\documentclass{article}
\usepackage{hyperref}
\usepackage{mla13}

\sources{metamorphosis}

\firstname{Wyatt}
\lastname{Johnson}
\professor{Dr. Melissa Stephens}
\class{English 123 X50}

\title{The Role of the Family Unit in Dealing with Change in
\citeauthor{kafka2007meta}'s \citetitle{kafka2007meta}}

% TOTAL WORD COUNT REQUIRED: 1500 % WORTH: 40% course grade

\begin{document}

\makeheader

A critical analysis of Franz Kafka's \citetitle{kafka2007meta} yields an
interesting story of a family and their struggles dealing with an affliction
to the eldest son, Gregor. Gregor's transformation into a cockroach is not
necessarily a disease but does exhibit a similar situation that could
befallen anyone suffering from any type of mental or physical illness
\cite[267]{rowe2002}. The Samsa family has to make a choice between Gregor,
who increasingly becomes a burden on the family, and their own well being.
Following his transformation, Gregor changes from being the bread winner in
the family to a burden on the family and becomes increasingly isolated as a
result of this change. Familial interaction is important to ones health,
which can be illustrated by Gregor's sister Grete. Grete changes from being
his loving sister, to his caregiver, and finally replacing him as the head
of the family. I will explore the struggle of Gregor's sister Grete and her
transformation as she struggles with Gregor's condition and resulting
becomes isolation. This raises a critical issue regarding the role the
family unit plays in the change and recovery of someone who is suffering as
Gregor has. The exploration of Grete's transformation and her interaction
with Gregor illustrates the transition that Gregor makes into his isolation,
depression, and finally death. Grete's perspective of her brother and role
in the family changes quite drastically throughout \citetitle{kafka2007meta}.

%%% BEGINING <- Protect and good nature

Grete shows her good nature towards Gregor at the beginning of this novella
by helping him through his new transformed life. She assumes the role almost
immediately as Gregor's caregiver by tending to his food needs and the
cleaning duties of his room. She went and left a dish full of milk and bread
for him to eat \cite[106]{kafka2007meta}. This kind thought could have been
motivated by the thinking that Gregor was still a human being stuck in a
cockroach body; thus it made sense that he would enjoy his previous
favourite food when he was in human form. In this context, we can interpret
the bread and milk as a symbol for the sympathy that the family feels for
Gregor's situation. On the one hand, the family is irritated that he had
become a cockroach, and the other they partially understand that this wasn't
his choice. The families situation can be seen clearly in the words of the
Chief Clerk, ``you're neglecting your official duties''
\cite[96]{kafka2007meta} as Gregor was previously the bread winner for the
family helping his father and family out of debt. The actions of his sister
at the beginning do reveal that they are indeed concerned at some level.
After it became apparent to his sister that he had not drank any of the milk
and bread, she went and took the effort to prepare an alternative meal. This
meal consisted primarily of garbage \cite[109]{kafka2007meta}. The garbage,
while heralded as exactly what was needed by Gregor, can be symbolized as
the family's loss of interest in his struggle, and thereby lack of
compassion. This transition utilizing the highlighted symbolism can be seen
as a foreshadowing for how his relationship with his sister and family
develops. She seemed alright with the idea that every time that Gregor
needed to be cared for, he was out of direct view. His sister seemed to want
to make it more comfortable for Gregor by cleaning the room and helping him
out, and ``for her part, [Grete] clearly sought to blur the embarrassment of
the whole thing'' \cite[115]{kafka2007meta}. \citeauthor{rowe2002} does
suggest, however, that if Grete had ``demonstrated in her demeanor and actions
that she still believed there was a human being underneath the insect''
\cite[277]{rowe2002}, it may have reduced the stress on Gregor. Just the
simple act of ``[w]itnessing'' his struggle, as the only person that even
seems to watch his progress is the charwoman. It is clear however that the
charwoman's presence annoys Gregor as after a period of her visiting him,
he made to attack her \cite[132]{kafka2007meta}. Grete's role in the
household began to take a larger piece of the family unit as she filled the
role of the leader in the family, she became less inclined to maintain the
work that she had once done to keep Gregor's room clean.

%%% MIDDLE <- Ignore and neglect

As the permanent nature of Gregor's condition sets in with the family, the
care from his sister deteriorates and begins to instigate a negative
reaction in Gregor. This deterioration as described by \citeauthor{rowe2002}
as ``Moral exhaustion,'' which describes the ``loss, however temporary, of
one's normal caring and empathetic attitudes and reactions to the ill
person'' and poses as the main ``threat to the patients humanity''
\cite[275, 275, 275]{rowe2002}. This transition was in part due to his
sister having to start to work to make up for the fact that Gregor was no
longer bringing in any income, leaving ``his sister, exhausted by office
work, [and] no longer had it in her to care for Gregor as she had done
earlier'' \cite[131]{kafka2007meta}. She never even talked about Gregor as
she once did. Her lack of remarks on him ``[having] a good appetite today''
or how the food had ``hardly been touched'' \cite[111, 111]{kafka2007meta}
were noticed by Gregor. The more that she realized that Gregor was no longer
the same person (or species), he began to frighten her just by the sight of
him. When she caught him unawares, she became frightened enough to ``[leap]
back and [lock] the door'' \cite[116]{kafka2007meta}. As mentioned in
\citeauthor{rowe2002}'s discussion on negative feelings, the caregiver
should ``keep those negative feelings from provoking acts of cruelty or
neglect'' \cite[265]{rowe2002} which in this case, is exactly what Grete is
doing. Her attitude towards Gregor continued however to change over time as
the foreshadowing suggests and it becomes clear that she is not as concerned
with him as she once was. His sister was ``[n]o longer bothering to think
what might please Gregor, his sister [...] now hurriedly shoved some food or
other into Gregor's room with her foot'' \cite[130]{kafka2007meta}. If we
take a step back however, \citeauthor{cantrellFamily} suggests that it may
be hard to see exactly that Gregor is still Gregor, given that from the
families perspective, he is just ``a gigantic insect who is incapable of
work'' \cite[581]{cantrellFamily}. This seems to come back to the idea of
``Memory'' as suggested by \citeauthor{rowe2002}. His family never really
saw much of him prior to his transformation, and as a result, they have very
little to work with in order to justify their efforts \cite[277]{rowe2002}.
Even the simple act of him crawling on a wall within view of his
mother resulted in another fainting spell, and stirred his sister to exclaim
````Ooh, Gregor!'' [...] brandishing her fist and glowering at him''
\cite[122]{kafka2007meta} as if his mother's fainting was his fault
directly. These actions led up to his depression that eventually
resulted in his death.

%%% FINISH, talk about leading up to his death

Up to Gregor's demise, there were periods of neglect from his family. At the
point when Gregor's sister suggests to move the furniture out of his room,
\citeauthor{cantrellFamily} notes that Grete is ``virtually nameless'' and
only being referred to as ``Gregor's sister'' \cite[584,
584]{cantrellFamily}. At this point, Grete has already established some
power in the family, and we are faced with some obvious imagery such as
after Gregor came to her when she was playing the violin and sat with her
hand placed on her fathers neck \cite[140]{kafka2007meta}. She even became
confident enough in her new role to suggest that ``[w]e must try and get rid
of [Gregor]'' \cite[138]{kafka2007meta}. Grete had become tired of dealing
with the troubles of her brother. Her attempt to rationalize this decision
comes from her justification that they had already done ``as much as humanly
possible to try and look after it and tolerate it'' and that ``[y]ou just
have to put it from your mind any thought that it's Gregor'' \cite[138,
139]{kafka2007meta}. Her parents seem resistant to the idea that they should
get rid of Gregor however. Her father then turns to Grete for what they
should do with him \cite[139]{kafka2007meta}. Their primary issue was that
they thought that Gregor could not understand them, and as a result, they
couldn't come ``to some sort of settlement with him''
\cite[139]{kafka2007meta}. This highlights the communication barrier that
was established after the metamorphosis. The ``fear of shame, habit of
secrecy'' \cite[585]{cantrellFamily} that surrounds the family prevents them
from seeking help from someone outside the family, leaving Gregor in the
sole hands of his family. Eventually, Gregor decided that he was indeed
causing too much trouble for his family. Motivated by the fact that ``[h]is
conviction that he needed to disappear was, if anything, still firmer than
his sister's'' \cite[141]{kafka2007meta}, Gregor died of self-inflicted
starvation.

%%% CONCLUSION

What we can determine from the flow of the plot-line leading up to Gregor's
suicide and his sisters involvement was that his family played a heavy role
in motivating his decision. Her good nature showed through with her care of
Gregor. Human nature stepped in however with the recognition that their
misfortunes were of greater need of their attention than it was to give
sympathy to Gregor. \citeauthor{kafka2007meta}'s \citetitle{kafka2007meta}
explores the role of the family unit in times of familial instability. It
can be concluded that the family unit plays an important role in dealing
with illness.

\makeworkscited

\end{document}
