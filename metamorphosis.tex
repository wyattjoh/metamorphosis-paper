\documentclass{article}
\usepackage{hyperref}
\usepackage{mla13}

\firstname{Wyatt}
\lastname{Johnson}
\professor{Dr. Melissa Stephens}
\class{English 123 X50: Literature in Global Perspective}
\title{The Role of the Family Unit in Dealing with Change}

\sources{metamorphosis}

\begin{document}
\makeheader

% PARAGRAPH ONE OF THE PROPOSAL SHOULD:

% - Identify the name of the author and the title of the major text to be studied ( choose either Kafka OR Cliff)
% - Identify the key topic or issue to be explored in the context of the text
% - Include a thesis statement (1-2 sentences in length). This statement offers an original and contestable claim (e.g., not the sky is blue, but a claim which is debatable and which requires supporting evidence from the main text). This claim should not simply replicate one made in the secondary literary criticism you have read
% - Provide justification for this thesis (e.g., suggest why you are pursuing it)
% - Indicate, generally, how you will support the thesis (methodology)
% - The first paragraph must be written in complete sentences

% PARAGRAPH TWO OF THE PROPOSAL SHALL BE THE OUTLINE AND SHALL:

% - Sketch a summary of how you plan to structure the essay
% - Identify KEY CLAIMS that you plan to make
% - Identify PRIMARY SUPPORTING EVIDENCE (e.g., examples of evidence).
% - Provide a few key quotations with page references.

A strict analysis of Franz Kafka's \citetitle{kafka2007meta} yields an
interesting story of a family, and their struggles dealing with Gregor's
transformation into a cockroach. Specifically, the Samsa family has to make
a choice in between Gregor, and themselves. As a result of the
transformation into a bug, Gregor is subjected to the abuses of isolation
from the family unit because he has now become an undesirable character that
the rest of the family and others looks poorly upon. I will explore the
struggle of each of the family members in order to highlight the piece-wise
transition into complete isolation. This raises the issue regarding the
importance of interaction within the context of this novella. Isolation from
one's family unit has the potential to be extremely damaging to ones health.

% Gregor's sister, Grete, shows great concern at the begining of the novel,
% but her reason is unclear. When Gregor wouldn't come out of his room, he
% concerned of why ``was she crying? Because he wouldn't get up and admit the
% chief clerk, because he was in danger of losing his job''
% \cite[95-96]{kafka2007meta}. At this point, it seemed like Gregor was
% thinking that his sister was more concerned for him losing his job rather
% than for himself. Even the chief clerk declared that he was just causing his
% parents ``grave and needless anxiety'' \cite[96]{kafka2007meta}.

Starting with Gregor's sister, Grete, I will focus on her transformation.
Grete, from her well-being nature of going through the effort of bringing him
different food to find what he would like. Even Gregor was surprised by this,
exclaiming that ``never would he have been able to guess what in the
goodness of her heart his sister did'' \cite[109]{kafka2007meta}. This then
degrading to her ``[n]o longer bothering to think what might please Gregor,
his sister [...] now hurriedly shoved some food or other into Gregor's room
with her foot'' \cite[130]{kafka2007meta}. We can also look at her attitude
discussing Gregor with the rest of the family. Often discussing the remark
relating to his food consumption as ``[having] a good appetite today'' or
``hardly been touched today'' \cite[111, 111]{kafka2007meta}. Discussing the
remainder of the characters in the family as well will prove useful.

Using Carol Cantrell's article \citetitle{cantrellFamily}, I will focus on
the change in the family structure as a result of Gregor's transformation.
His sense of duty is made clear right at the beginning of the book,
\citeauthor{cantrellFamily} remarking that ``his desire to sleep a little
longer is an implicit threat to the rest of the family''
\cite[581]{cantrellFamily}. The analysis by \citeauthor{cantrellFamily}
regarding Gregor's view of the family, and how Gregor becomes ``free to
think ill or even indifferently of his family'' \cite[585]{cantrellFamily}
because he is dying.

With the use of Michael Rowe's \citetitle{rowe2002}, I will focus on the
relationships drawn with how Gregor's now bug state has changed his families
view of him.

\makeworkscited
\end{document}
