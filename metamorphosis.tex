\documentclass{article}
\usepackage{hyperref}
\usepackage{mla13}

\sources{metamorphosis}

\firstname{Wyatt}
\lastname{Johnson}
\professor{Dr. Melissa Stephens}
\class{English 123 X50}
\title{The Role of the Family Unit in Dealing with Change in
\citeauthor{kafka2007meta}'s \citetitle{kafka2007meta}}

\newcommand{\comment}[1]{}

% TOTAL WORD COUNT REQUIRED: 1500
% WORTH: 40% course grade

\begin{document}
\makeheader
% PARAGRAPH ONE OF THE PROPOSAL SHOULD:

% - Identify the name of the author and the title of the major text to be studied ( choose either Kafka OR Cliff)
% - Identify the key topic or issue to be explored in the context of the text
% - Include a thesis statement (1-2 sentences in length). This statement offers an original and contestable claim (e.g., not the sky is blue, but a claim which is debatable and which requires supporting evidence from the main text). This claim should not simply replicate one made in the secondary literary criticism you have read
% - Provide justification for this thesis (e.g., suggest why you are pursuing it)
% - Indicate, generally, how you will support the thesis (methodology)
% - The first paragraph must be written in complete sentences

% PARAGRAPH TWO OF THE PROPOSAL SHALL BE THE OUTLINE AND SHALL:

% - Sketch a summary of how you plan to structure the essay
% - Identify KEY CLAIMS that you plan to make
% - Identify PRIMARY SUPPORTING EVIDENCE (e.g., examples of evidence).
% - Provide a few key quotations with page references.

A critical analysis of Franz Kafka's \citetitle{kafka2007meta} yields an
interesting story of a family and their struggles dealing with an affliction
to the eldest son, Gregor. Gregor's transformation into a cockroach is not
necessarily a disease, this does result in a similar situation that could
result to anyone suffering from any type of mental or physical illness
\cite[267]{rowe2002}. The Samsa family has to make a choice between Gregor,
who increasingly becomes a burden on the family, and their own well being.
Following his transformation, Gregor changes from being the bread winner in
the family to a burden on the family and becomes increasingly isolated as a
result of this change. Familial interaction is important to ones health,
which can be illustrated by Gregor's sister Grete. Grete changes from being
his loving sister, to his caregiver, and finally replacing him as the head
of the family. I will explore the struggle of Gregor's sister Grete and her
transformation as she struggles with Gregor's condition and resulting
becomes isolation. This raises a critical issue regarding the role the
family unit plays in the change and recovery of someone who is suffering as
Gregor has. The exploration of Grete's transformation and her interaction
with Gregor illustrates the transition that Gregor makes into his isolation,
depression, and finally death.

Grete's perspective of her brother and role in the family changes quite
drastically throughout \citetitle{kafka2007meta}. Grete shows her good
nature towards Gregor at the beginning by helping him through his new
transformed life. Grete assumes the role almost immediately as Gregor's 
caregiver, tending to his food and the cleanliness of his room. After the
initial discovery that Gregor had transformed and things had calmed down a
bit Grete left a dish full of milk and bread for him to eat. This idea could
be motivated by the thinking that as the cockroach was still Gregor to some
level that he would enjoy his previous favourite food. In this context, we
can interpret the bread and milk as a symbol for the sympathy that the
family feels for Gregor's condition. After it became apparent that Gregor
did not like the food that he once did as a human, she went and took the
effort to prepare an alternative meal for him which consisted primarily of
garbage. The garbage, while heralded as exactly what was needed by Gregor,
could be symbolized as the family's loss of interest in his struggle. This
transition can be seen as a foreshadowing as to how his relationship with
his sister develops as she becomes the head of the household. As this
transition occurs with Grete, it became more and more of a task for her to
maintain the same level of care that Gregor had grown to expect. This was in
part with Grete having to start to work to make up for the fact that Gregor
was no longer bringing in any income, leaving ``his sister, exhausted by
office work, [and] no longer had it in her to care for Gregor as she had
done earlier'' \cite[131]{kafka2007meta}. She never even talked about Gregor
as she once did, even the simple lack of remarks on him ``[having] a good
appetite today'' or how the food had ``hardly been touched'' \cite[111,
111]{kafka2007meta} were noticed by Gregor. There were lots of opportunities
at the start for Grete to complain about her predicament, but for the most
part she kept her cool. She seemed alright with the idea that every time
that Gregor needed to be cared for, he was out of direct view. His sister
seemed to want to make it more comfortable for Gregor by cleaning the room
and helping him out, and ``for her part, [Grete] clearly sought to bluer the
embarrassment of the whole thing'' \cite[115]{kafka2007meta}. It was clear
that she was embarrassed of Gregor, and his appearance. The more that she
realized that Gregor was no longer the same person (or species), he began to
frighten her just by the sight of him. When she caught him unawares, she
became frightened enough to ``[leap] back and [lock] the door''
\cite[116]{kafka2007meta}. As mentioned in \citeauthor{rowe2002}'s
discussion on negative feelings, the caregiver should ``keep those negative
feelings from provoking acts of cruelty or neglect'' \cite[265]{rowe2002}
which in this case, is exactly what Grete is doing. Her attitude towards
Gregor continued to change over time as the foreshadowing suggests and it
becomes clear that she is not as concerned with him as she once was. His
sister was ``[n]o longer bothering to think what might please Gregor, his
sister [...] now hurriedly shoved some food or other into Gregor's room with
her foot'' \cite[130]{kafka2007meta}.

% % WHOLE PARAGRAPH NEEDS WORK
% Using Carol Cantrell's article \citetitle{cantrellFamily}, I will focus on the change in the family structure as a result of Gregor's transformation. His sense of duty is made clear right at the beginning of the book, \citeauthor{cantrellFamily} remarking that ``his desire to sleep a little longer is an implicit threat to the rest of the family'' \cite[581]{cantrellFamily}. The analysis by \citeauthor{cantrellFamily} regarding Gregor's view of the family, and how Gregor becomes ``free to think ill or even indifferently of his family'' \cite[585]{cantrellFamily} because he is dying.
%
% % EXPAND EXPAND EXPAND!
% With the use of Michael Rowe's \citetitle{rowe2002}, I will focus on the relationships drawn with how Gregor's now bug state has changed his families view of him.

\makeworkscited
\end{document}
